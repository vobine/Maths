\documentclass[twocolumn]{article}

\usepackage{amsmath}
\usepackage{amssymb}

\usepackage{amsthm}
\newtheorem{definition}{Definition}
\newtheorem{lemma}{Lemma}
\newtheorem{theorem}{Theorem}

\usepackage{hyperref}

\title{Maximum Partition-Product}
\author{Hal Peterson}

\begin{document}

\maketitle

\section{Introduction}

\begin{definition}
  \label{definition:partition}
  A partition $P(n)$ of an integer $n$ is a multiset of integers with elements, $p_i \in P(n)$ that sum to $n$.
  All $p_i \geq 1$.
\end{definition}

For example, two distinct partitions of 6 are: $P(6) = \{ 2, 2, 2 \}$ and $P(6) = \{ 1, 2, 3 \}$.  There are many more.

\begin{definition}
  \label{definition:partition-product}
  A partition-product $P^*$ is the product of all elements of a partition.
\end{definition}

For example, $P^*(\{ 1, 2, 3 \}) = 6$.

The partition-product problem is, given $n$, to find a partition $P(n)$ to maximize the partition-product.

For example, the distinct partitions of $n=4$ are:
\begin{eqnarray*}
  P(4) & = & \{ 4 \} \\
  & = & \{ 1, 3 \} \\
  & = & \{ 2, 2 \} \\
  & = & \{ 1, 1, 2 \} \\
  & = & \{ 1, 1, 1, 1 \}
\end{eqnarray*}
Their partition-products are, respectively, $4, 3, 4, 2, 1$.  So the maximum partition-product is
\begin{equation*}
  P^*(\{ 4 \}) = P^*(\{ 2, 2 \}) = 4.
\end{equation*}

\section{Solution(s)}

Let's start with three observations.

\subsection{Smaller is Better}

\subsection{But $1$ is a Waste}

\subsection{And $2$ is too Small}

\subsection{Sub-Partitions}



\subsection{General Solution}

\begin{theorem}
  Given $n$, the maximum partition-product $\max{P^*(P(n))}$
\end{theorem}

\section{Sources}

The original idea came from the thumbnail description of a YouTube video on Presh Talkwakar's channel Mind Your Decisions:  \url{https://youtu.be/o-UF8JlqZ_c}.  I have not watched the video yet.

The thumbnail refers to a problem from the 1976 International Mathematical Olympiad:  \url{https://www.imo-official.org/problems.aspx}.  The problem is:
\begin{quote}
  Determine, with proof, the largest number which is the product of positive integers whose sum is 1976.
\end{quote}

\end{document}
