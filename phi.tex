\documentclass{article}

\usepackage{amsmath}
\usepackage{amssymb}
\usepackage{graphicx}

\title{Powers of the Golden Ratio}
\author{Hal Peterson}

\begin{document}
\maketitle

\begin{figure}
  \fbox{\includegraphics[width=8cm]{Phi3.pdf}}
  \caption{Defining $\phi$ as a ratio of lengths (NTS)}
  \label{fig:phi}
\end{figure}
A definition of the golden ratio $\phi$ starts in Figure~\ref{fig:phi}, with two line segments cunningly designed so that the ratio of their lengths, $B/A$,
is equal to the ratio of the length of their sum to the longer segment:
\begin{equation}
  B/A = (A+B)/B.
  \label{phi:ratios}
\end{equation}
We can juggle that a bit to find something to solve:
\begin{eqnarray*}
  B/A & = & (A + B)/B \\
  B^2 & = & A (A + B) \\
  & = & A^2 + AB.
\end{eqnarray*}
We seek the ratio $B/A$; we can simplify the equation by setting $A=1$.
Then the ratio is simply $B$ and the equation becomes:
\begin{equation*}
  B^2 = 1^2 + 1B = 1 + B.
\end{equation*}
So, if $\phi$ exists, it is a positive solution for $x$ in:
\begin{equation}
  x^2 - x - 1 = 0.
  \label{phi:quadratic}
\end{equation}

Solve by completing the square.
\begin{eqnarray*}
  0 & = & x^2 - x - 1 \\
  & = & (x^2 - x + 1/4) - 5/4 \\
  & = & (x - 1/2)^2 - 5/4 \\
  5/4 & = & (x - 1/2)^2 \\
  \pm\sqrt{5/4} & = & x - 1/2 \\
  \frac{1 \pm \sqrt{5}}{2} & = & x.
\end{eqnarray*}

Because $(1-\sqrt{5})/2 < 0$, the answer must be the other root:
\begin{equation}
  \phi = \frac{1 + \sqrt{5}}{2}.
  \label{phi:value}
\end{equation}
Is it really?  Verify by substituting $\phi$ from Equation~\ref{phi:value} for $x$ in Equation~\ref{phi:quadratic}.
\begin{eqnarray*}
  0 & = & x^2 - x - 1 \\
  & = & \phi^2 - \phi - 1 \\
  & = & \left(\frac{1 + \sqrt{5}}{2}\right)^2 - \frac{1 + \sqrt{5}}{2} - 1 \\
  & = & \frac{1 + 2\sqrt{5} + 5}{4} - \frac{2 + 2\sqrt{5}}{4} - 1 \\
  & = & \frac{1 + 5 - 2 + 2\sqrt{5} - 2\sqrt{5}}{4} - 1 \\
  & = & 4/4 - 1 \\
  & = & 0. \hspace{20pt}\blacksquare
\end{eqnarray*}
Yes!

Now for the powers.  Watch the coefficients:
\begin{eqnarray*}
  \phi^1 & = & \phi \\
  \phi^2 & = & \phi + 1 \\
  \phi^3 & = & \phi(\phi + 1) = \phi^2 + \phi = \phi + 1 + \phi = 2\phi + 1 \\
  \phi^4 & = & \phi(2\phi + 1) = 2\phi^2 + \phi = 2(\phi + 1) + \phi = 3\phi + 2 \\
  \phi^5 & = & \phi(3\phi + 2) = 3\phi^2 + 2\phi = 3(\phi + 1) + 2\phi = 5\phi + 3 \\
  \phi^6 & = & \phi(5\phi + 3) = 5\phi^2 + 3\phi = 5(\phi + 1) + 3\phi = 8\phi + 5 \\
  & & \ldots.
\end{eqnarray*}

The coefficients of the $\phi$ term are $\langle 1, 1, 2, 3, 5, 8, \ldots\rangle$.  The constant terms are $\langle 0, 1, 1, 2, 3, 5, \ldots\rangle$.  These sequences look suspiciously similar and suspiciously familiar.

Generalize for $\phi^n$:
\begin{equation}
  \phi^n = a_n\phi + b_n
\end{equation}

Then:
\begin{eqnarray*}
  \phi^{n+1} & = & a_{n+1}\phi + b_{n+1} \\
  & = & \phi(a_n\phi + b_n) \\
  & = & a_n\phi^2 + b_n\phi \\
  & = & a_n(\phi + 1) + b_n\phi \\
  & = & (a_n + b_n) \phi + a_n.
\end{eqnarray*}
With a shift of indices:
\begin{eqnarray*}
  b_n & = & a_{n-1} \\
  a_{n+1} & = & a_n + b_n \\
  & = & a_n + a_{n-1}.
\end{eqnarray*}
That is, indeed, the Fibonacci sequence in the coefficients of the $\phi$ terms.  The constant terms are the same Fibonacci sequence lagged by one.

\end{document}
