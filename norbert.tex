\documentclass[twocolumn]{article}
\pagestyle{headings}

\usepackage{amsmath}
\usepackage{amssymb}

\usepackage{amsthm}
\theoremstyle{definition}
\newtheorem{definition}{Definition}
\theoremstyle{plain}
\newtheorem{lemma}{Lemma}
\newtheorem{theorem}{Theorem}

% \usepackage{hyperref}

\usepackage{graphicx}

\usepackage{cite}

\title{Norbert Wiener's Elevator Problem: \\ Pi Day 2025}

\author{Hal Peterson}

\begin{document}

\maketitle

\section{A Norbert Wiener Story}

There are lots of Norbert Wiener stories~\cite{Hardesty2021}.  In one
of them, a colleague 
set a question for Wiener as the two of them waited for an elevator;
Wiener announced the solution as the elevator reached the fourth
floor.  The problem:
\begin{quotation}
  You have a dowel, two inches in diameter.  Use a two-inch diameter
  bit to drill a hole through the dowel, perpendicular to the axis and
  straight through the center.  What is the volume of the wood removed
  from the dowel?
\end{quotation}

\begin{figure}
  \includegraphics[scale=0.25]{norbert.png}
  \caption{Overview:  intersecting cylinders.}
  \label{fig:overview}
\end{figure}

See Figure~\ref{fig:overview} for an illustration:  call the yellow
cylinder the dowel and the green cylinder the drill bit.\footnote{Vice
versa would work as well, since the problem is symmetric.}

\begin{figure}
  \includegraphics[scale=0.25]{norbert-result.png}
  \caption{Boundary of the volume removed by the drill.}
  \label{fig:boundary}
\end{figure}

To illustrate the symmetry, see Figure~\ref{fig:boundary}, showing the
intersection, with the rest of the two cylinders removed.  Note
also the apparent color swap:  the left side, where the yellow
cylinder was in Figure~\ref{fig:overview}, is now green.  Why?

The colors did not change.  The green bits in
Figure~\ref{fig:boundary} are part of the surface of the green
cylinder that were covered up by the yellow cylinder in
Figure~\ref{fig:overview}.  \emph{All} of the yellow and green
surfaces in Figure~\ref{fig:boundary} belong the surface of one of the
two original cylinders.  That fact will be useful in computing the
volume.

\section{Geometry}

As is clear from Figure~\ref{fig:boundary}, the drilled-out volume is
not one of the easy standard shapes, e,g, a sphere.  So, what
\emph{is} it?

Let's build it from its components.  First, what is a cylinder?  Let's
define it as the set of points (\emph{locus}, if you want the jargon)
at a constant perpendicular distance, $r$, from a line in 3-D space.
So for a cylinder centered on, say, the $\hat{x}$ axis, then the
cylinder satisfies
\begin{equation}
  y^2 + z^2 = r^2.
\end{equation}
\label{eq:cylinder}
For any given value of $x$, this defines a circle; over all values of
$x$ it's a cylinder, with circular cross-section.
This becomes clearer in orthogonal projections of our intersecting
cylinders onto the $xy$, $yz$, and $xz$ planes, shown in
Figure~\ref{fig:projections}.
\begin{figure}
  \includegraphics[scale=0.5]{norbert-projection.png}
  \caption{Three plane projections of intersecting cylinders.  Upper
    left:  looking down the $\hat{z}$-axis; upper right:  looking down
    the $\hat{x}$-axis; lower left:  looking along the
    $\hat{y}$-axis.  The dashed square on the upper left projection is
    explained in the text.}
  \label{fig:projections}
\end{figure}

What about the intersection of two perpendicular cylinders with equal
radii, as in Figure~\ref{fig:overview}?  Let's assume that the green
cylinder is centered on the $\hat{x}$ axis and the yellow cylinder is
centered on the $\hat{y}$ axis.  Then their respective equations are:
\begin{eqnarray*}
  x^2 & = & r^2 - z^2 \\
  y^2 & = & r^2 - z^2,
\end{eqnarray*}
or:
\begin{eqnarray}
  x(z) & = & \pm \sqrt{r^2 - z^2} \label{eq:x(z)} \\
  y(z) & = & \pm \sqrt{r^2 - z^2}. \label{eq:y(z)}
\end{eqnarray}

Notice that $x(z) = y(z)$ for any given value of $z$.  That implies
that the plane figure is a square.\footnote{This needs support, no?}

\begin{figure}
  \includegraphics[scale=0.25]{norbert-square.png}
  \caption{The blue patch, corresponding to the dashed box in
    Figure~\ref{fig:projections}, is a square.}
  \label{fig:square}
\end{figure}

Consider, in Figure~\ref{fig:square}, the intersection with everything
above $z=0.75$ planed off.  The blue shape is a square.\footnote{Prove it!}

\section{Computing the Volume}
\label{sec:computing}

We're going to compute the volume $V$ of the octant of the
intersection with all nonnegative coordinates.  Because the
intersection is symmetrical, the volume of the whole is $8V$.

We'll integrate a stack of squares, each in a plane normal to the
$\hat{z}$ axis, starting with $z=0$ and proceeding to $z=r$.

One corner of each square is on the $\hat{z}$ axis, at $x=y=0$.  The
other corner is at the positive-$\hat{x},\hat{y}$ intersection of the
surfaces of the two cylinders.  For now we'll leave those coordinates
as a function of $z$:

\begin{equation*}
  V = \int_{z=0}^r x(z) y(z) dz.
\end{equation*}

We already have expressions for $x(z)$ and $y(z)$ from
Equations~\ref{eq:x(z)} and~\ref{eq:y(z)}.  Substituting into the
equation for $V$:

\begin{eqnarray*}
  V & = & \int_{z=0}^r \sqrt{r^2 - z^2} \sqrt{r^2 - z^2} dz \\
  & = & \int_{z=0}^r r^2 - z^2 dz \\
  & = & \left. (zr^2 - z^3/3) \right|_{z=0}^r \\
  & = & (r^3 - r^3/3) - (0r^2 - 0^3/3) \\
  & = & 2r^3/3.
\end{eqnarray*}

As described above, $V$ covers one octant of the intersection, and
$r=1$, so the total intersection volume $T$ is:
\begin{eqnarray*}
  T & = & 8V \\
  & = & 8(2r^3/3) \\
  & = & 16(1^3)/3 \\
  & = & 16/3.
\end{eqnarray*}

\section{Coda}

Somehow the answer is always in the back of the book, if you look hard
enough~\cite{Wikipedia-Steinmetz}.

\bibliography{norbert}
\bibliographystyle{plain}

\end{document}
