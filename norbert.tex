\documentclass[twocolumn]{article}

\usepackage{amsmath}
\usepackage{amssymb}

\usepackage{amsthm}
\theoremstyle{definition}
\newtheorem{definition}{Definition}
\theoremstyle{plain}
\newtheorem{lemma}{Lemma}
\newtheorem{theorem}{Theorem}

\usepackage{hyperref}

\usepackage{graphicx}

\usepackage{cite}

\title{Norbert Wiener's Elevator Problem}

\author{Hal Peterson}

\begin{document}

\maketitle

\section{The Norbert Wiener Story}

There are lots of Norbert Wiener stories \cite{Hardesty2021}.  In one
of them, a colleague 
set a question for Wiener as the two of them waited for an elevator;
Wiener announced the solution as the elevator reached the fourth
floor.  The problem:
\begin{quotation}
  You have a dowel, two inches in diameter.  Use a two-inch diameter
  bit to drill a hole through the dowel, perpendicular to the axis and
  straight through the center.  What is the volume of the wood removed
  from the dowel?
\end{quotation}

\begin{figure}
  \includegraphics[scale=0.25]{norbert.png}
  \caption{Overview:  intersecting cylinders.}
  \label{fig:overview}
\end{figure}

See Figure~\ref{fig:overview} for an illustration:  call the yellow
cylinder the dowel and the green cylinder the drill bit (or vice
versa, the problem is symmetric).

\begin{figure}
  \includegraphics[scale=0.25]{norbert-result.png}
  \caption{Boundary of the drilled-out volume.}
  \label{fig:boundary}
\end{figure}

To see the symmetry, Figure~\ref{fig:boundary} shows the volume
drilled out.

\section{Geometry}

\section{Computing the Volume}

\bibliography{norbert}
\bibliographystyle{plain}

\end{document}
